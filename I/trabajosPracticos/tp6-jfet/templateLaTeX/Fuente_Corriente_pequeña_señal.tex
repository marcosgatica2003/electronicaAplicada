\section{Cálculos analíticos y experimentales de Zi, Zo, Av y Ai.} 
% poner fotossss

\imagen[Circuito para primera parte de pequeña señal]{15cm}{./imagenes/impeentrada.png}

\sangria{} Para hacer las mediciones de la impedancia de entrada, ganancia de tensión y ganancia de corriente se inserta la señal de $1kHz$ por el capacitor de acoplamiento en la base del transistor $C_i$ y se va aumentando la tensión de la señal de entrada hasta tener una tensión de salida de $V_L=1V_{pp}$.
\sangria{} Para hacer estas mediciones se se agrega una resistencia en serie en la entrada de la base denominada resistencia sensora $R_S=1234567$ Donde se mide la tensión $V_g$ y $V_i$ definida en la figura 1.
\midTitle{blue}{Mediciones Obtenidas}{}

\begin{itemize}
    \item ola Rao
\end{itemize}