% \newpage
\section{Polarización Con fuente de corriente}
\subsection{Cálculo de $R_1$}
\midTitle{red}{Parte analítica}{}
\sangria{En la segunda parte del trabajo práctico, se reemplaza la resistencia de drenador $R_D$ por una \textbf{carga activa}. Como se vio en la teoría, en los circuitos integrados se utilizan fuentes de corriente de transistores como elemento de carga, ya que son menos sensibles a las variaciones de temperatura y ocupan una menor superficie que los resistores pasivos.}

\sangria{El circuito a implementar es una \textbf{Fuente de Corriente Básica} con BJT, también conocida como espejo de corriente. Esta configuración utiliza dos transistores (Q1 y Q2) apareados, donde Q1 se conecta como un diodo para establecer una corriente de referencia $I_R$.}



\begin{figure}[!ht]
\centering
\resizebox{0.3\textwidth}{!}{%
\begin{circuitikz}
\tikzstyle{every node}=[font=\tiny]
\draw (5.5,11.25) to[Tpnp, transistors/scale=1.19] (5.5,13.25);
\draw (3.5,11.25) to[Tpnp, transistors/scale=1.19] (3.5,13.25);
\draw (3,11.25) to[short] (4.25,11.25);
\draw (4.25,11.25) to[short] (4.25,12.25);
\draw (3,11.25) to[short] (2.25,11.25);
\draw (2.25,11.25) to[R] (1.25,11.25);
\draw (0.75,11) to (0.75,10.75) node[sground]{};
\draw (1.25,11.25) to[short] (0.75,11.25);
\draw (0.75,11.25) to[short] (0.75,10.75);
\draw (7.25,13.25) to[short] (7.25,11.5);
\draw (7.25,11.5) to[american voltage source] (7.25,8.25);
\draw (2.5,10.25) to[sinusoidal voltage source, sources/symbol/rotate=auto] (2.5,7.75);
\draw (4.25,10.25) to[curved capacitor] (2.5,10.25);
\draw (4.25,10.25) to[R] (4.25,7.75);
\draw (2.5,7.75) to[short] (7.25,7.75);
\draw (7.25,8.25) to[short] (7.25,7.75);
\draw [->, >=Stealth] (4.25,10.25) -- (5,10.25);
\draw [short] (5,10.5) -- (5,10);
\draw [short] (5,10.75) -- (5,10.5);
\draw [short] (5,9.75) -- (5,10);
\draw [short] (5,10.75) -- (5.25,10.75);
\draw [short] (5,9.75) -- (5.25,9.75);
\draw [short] (5.5,11.25) -- (5.5,10.75);
\draw [short] (5.5,10.75) -- (5.25,10.75);
\draw (5.25,9.75) to[R] (5.25,7.75);
\draw (6,9.5) to[curved capacitor] (6,7.75);
\draw (5.25,9.75) to[short] (5.75,9.75);
\draw (5.5,11.25) to[short] (6.75,11.25);
\node [font=\normalsize] at (6.75,11.5) {+};
\node [font=\normalsize] at (6.5,10) {$V_{L}$};
\node [font=\normalsize] at (5.5,12.25) {Q2};
\node [font=\normalsize] at (3.5,12.25) {Q1};
\node [font=\normalsize] at (3.5,9.25) {$R_{G}$};
\node [font=\normalsize] at (4.75,8.75) {$R_{S}$};
\node [font=\normalsize] at (1.5,11.75) {$R_{1}$};
\draw [short] (2.5,12.25) -- (2.5,12.75);
\draw [short] (4.5,12.75) -- (4.5,12.25);
\draw [short] (3.5,13.25) -- (7.25,13.25);
\node at (3.5,11.25) [circ] {};
\node at (5.5,11.25) [circ] {};
\node at (5.5,13.25) [circ] {};
\node at (3.5,13.25) [circ] {};
\node [font=\Large] at (6.5,8) {$-$};
\draw (6,9.75) to[short] (6,9);
\draw (6,9.75) to[short] (5.5,9.75);
\draw (2.5,12.75) to[crossing] (4.5,12.75);
\draw (4.25,12.25) to[short] (4.75,12.25);
\end{circuitikz}
}%

\label{fig:my_label}
\end{figure}

\sangria{El objetivo de esta parte es \textbf{mantener el mismo punto Q} (la misma $I_{DQ}$ y $V_{DSQ}$) que en el diseño con carga resistiva. Por lo tanto, la corriente de salida $I_O = I_{C2}$ de nuestro espejo de corriente debe ser igual a la corriente $I_{DQ}$ que calculamos en la primera parte. Para lograr esto, se debe calcular la resistencia $R_1$ que establece la corriente de referencia $I_R$. Asumiendo un $\beta$ alto, la corriente de referencia $I_R$ es aproximadamente igual a la corriente de salida $I_{C2}$. Por lo tanto, despejamos $R_1$ de la ecuación de la malla de entrada, usando la corriente $I_{DQ}$ deseada:}

$$I_{R} = \frac{V_{DD}-V_{BE}}{R_{1}} \quad \xrightarrow{\quad \scalebox{1.2}{$\text{si } I_R \approx I_{C2} = I_{DQ}$} \quad} \quad R_{1} = \frac{V_{DD}-V_{BE}}{I_{DQ}}$$

$$R_1=\frac{12V-0,7V}{3,325mA}\quad \rightarrow \quad R_1=3200\Omega \quad \xrightarrow{\quad \scalebox{1.2}{\text{Normalizado} \quad}} \quad R_1=3300\Omega $$

\subsection{implementación circuito con carga activa}
\midTitle{red}{Parte práctico}{}
\sangria{Para la implementación de la carga activa, se procedió a calcular el valor de la resistencia $R_1$ para el espejo de corriente, con el objetivo de mantener el $I_{DQ}$ de diseño ($3.325\text{ mA}$). Sin embargo, la implementación práctica presentó serias dificultades. En un primer intento se utilizaron transistores PNP \textbf{MPSA92}, y posteriormente \textbf{A92 B331}. En ambos casos, el circuito falló: la señal de salida $v_L$ se observaba \textbf{severamente recortada}.}

\sangria{Este recorte sugiere que el punto Q del JFET se estaba desplazando incorrectamente, probablemente empujando al transistor a la región óhmica. Esto puede ocurrir si el espejo de corriente no logra establecer una corriente de colector ($I_{DQ}$) estable, un problema común si los transistores (como los modelos de alto voltaje MPSA92) no están caracterizados para operar eficientemente a corrientes tan bajas, presentando un $\beta$ muy bajo o un $V_{BE}$ inconsistente.}


\sangria{Finalmente, se implementó el espejo de corriente utilizando transistores \textbf{BC 557}, con los cuales el circuito funcionó correctamente. Se midieron los $\beta$ de ambos transistores, arrojando valores de \textbf{156 y 159}, lo cual es ideal para un espejo de corriente. Con esta configuración, se midió la corriente $I_{DQ}$ de forma indirecta, midiendo la caída de tensión en $R_S$ ($80.2\ \Omega$), la cual fue de $261.7\text{ mV}$, resultando en: $I_{DQ} = 3.263\text{ mA}$. El valor medido para $V_{DSQ}$ fue de \textbf{9.01\text{ V}}.}

\begin{table}[H]
\caption{Comparativa de los valores del punto Q}
\label{tab:comparativa-q} % ¡Recordá usar un label único!
\resizebox{\textwidth}{!})} & \textbf{Medido (Pasivo)} & \textbf{Medido (Activo)} & \textbf{Desvío (Pasivo / Activo)} \\ \hline

% Fila VDSQ
$V_{DSQ}$ & 6 V & $[5.4\text{ V} , 6.6\text{ V}]$ & 6.12 V & 9.01 V & 2.0\% / 50.17\% \\ \hline

% Fila IDQ
$I_{DQ}$ & 3.325 mA & $[2.99\text{ mA} , 3.66\text{ mA}]$ & 3.144 mA & 3.263 mA & 5.4\% / 1.86\% \\ \hline
\end{tabular}%
}
\end{table}

\midTitle{teal}{Resultados}{}

\sangria{La implementación de la carga activa fue exitosa tras reemplazar los transistores PNP por el modelo \textbf{BC 557}. Los resultados de la polarización son notables: la corriente $I_{DQ}$ medida con la carga activa ($3.263\text{ mA}$) es \textbf{extremadamente cercana} al valor teórico de diseño ($3.325\text{ mA}$), con un desvío de solo \textbf{1.86\%}. Esto valida el diseño del espejo de corriente. Sin embargo, la tensión $V_{DSQ}$ ($9.01\text{ V}$) se desvía significativamente del objetivo de $6\text{ V}$. Esto es esperable, ya que la resistencia de salida $r_o$ del espejo de corriente es mucho más alta que la $R_D$ pasiva, reubicando el punto Q verticalmente, lo cual tendrá un impacto directo en la ganancia de tensión (Av) del amplificador.}
