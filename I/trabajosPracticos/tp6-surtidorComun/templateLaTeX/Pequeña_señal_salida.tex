\subsection{Cálculo experimental de impedancia de salida}

\imagen[Circuito para medir impedancia de salida]{15cm}{./imagenes/pepe.png}

\sangria{Para medir la impedancia de salida se pasiva la fuente de señal de entrada ($V_i=0$) y se conecta una resistencia sensora ($R_{sensor} = \SI{1.8}{\kilo\ohm}$) en serie con el capacitor de acoplamiento. Se inserta una señal de $1\text{ kHz}$ por esta rama y se mide la tensión $V_g$ (antes de la $R_{sensor}$) y $V_o$ (en el drenador), buscando ajustar $V_o=1\text{ V}_{pp}$.}

\sangria{La impedancia de salida se calcula como:}
$$Z_o=\frac{V_o}{I_o}=\frac{V_o}{\frac{V_g-V_o}{R_{sensor}}}$$

\midTitle{blue}{Mediciones Obtenidas (Zo)}{}
\begin{itemize}
    \item $V_o=1\text{ V}_{pp}$
    \item $V_g=2,25\text{ V}_{pp}$
    \item $R_{sensor}=\SI{1.8}{\kilo\ohm}$
\end{itemize}

\sangria{Con los valores obtenidos en las mediciones se tiene el siguiente resultado:}
$$Z_o=\frac{1\text{ V}}{\frac{2,25\text{ V}-1\text{ V}}{1800\ \Omega}} \quad \rightarrow \quad Z_o = 1440\ \Omega$$

\sangria{Con este valor de impedancia de salida se puede calcular el valor de $r_{ds}$ del transistor:}
$$Z_o=r_{ds} \parallel R_D \quad \rightarrow \quad r_{ds}=\frac{Z_o \cdot R_D}{R_D-Z_o}$$
\sangria{Usamos el $R_D$ normalizado ($1.8\text{ k}\Omega$) y el $Z_o$ medido:}
$$r_{ds}=\frac{1440\ \Omega \cdot 1800\ \Omega}{1800\ \Omega-1440\ \Omega} \quad \rightarrow \quad r_{ds} = 7200\ \Omega$$

% --- TABLA COMPARATIVA FINAL ---
\begin{table}[H]
\caption{Comparativa de parámetros de Pequeña Señal (Calculado vs. Medido)}
\label{tab:comparativa-ac-final} % ¡Recordá usar un label único!
\resizebox{\textwidth}{!}{%
\begin{tabular}{|
>{\columncolor[HTML]{FFCCC9}}c |
c | c | c |}
\hline
\rowcolor[HTML]{FFFC9E}
\textbf{Parámetro} & \textbf{Valor Calculado (Analítico)} & \textbf{Valor Medido (Experimental)} & \textbf{Desvío Porcentual} \\ \hline

% Fila Zi
$Z_i$ & $\SI{1}{\mega\ohm}$ & $1,088\text{ M}\Omega$ & 8,8\% \\ \hline

% Fila Zo
$Z_o$ & $1681,51\ \Omega$ & $1440\ \Omega$ & 14,4\% \\ \hline

% Fila Av
$A_v$ & $20,62$ & $27,027$ & 31,1\% \\ \hline

% Fila Ai
$A_i$ & $12263$ & $16339$ & 33,2\% \\ \hline
\end{tabular}%
}
\end{table}

% --- SECCIÓN DE RESULTADOS ---
\midTitle{teal}{Resultados}{}

\sangria{Las mediciones experimentales de pequeña señal muestran una excelente correlación en la impedancia de entrada $Z_i$, con un desvío de solo \textbf{8.8\%} respecto al valor teórico de $\SI{1}{\mega\ohm}$. La impedancia de salida $Z_o$ también arrojó un valor ($1440\ \Omega$) cercano al esperado ($1681,51\ \Omega$), con un desvío del \textbf{14.4\%}. Esto nos permitió estimar la resistencia interna del JFET, $r_{ds}$, en $7.2\text{ k}\Omega$.}

\sangria{Las ganancias de tensión y corriente presentaron un desvío mayor ($31,1\%$ y $33,2\%$ respectivamente). Esto es esperable, ya que la ganancia depende directamente de $g_m$, un parámetro que varía significativamente con el punto Q real del circuito (el cual se ajustó con $R_S=80,2\ \Omega$ en la práctica) y con las características propias del transistor utilizado.}
