\subsection{Cálculo experimental de Zo}
\imagen[Circuito para mediciones de Zo]{15cm}{./imagenes/corrienteZo.png}
\imagen[Placa al realizar las mediciones para Zo]{15cm}{./imagenes/impesalida.jpg}
\sangria{} Para medir la impedancia de salida se pasiva la fuente de señal y se conecta una resistencia sensora $R_S=123456$ en serie con un el capacitor de acoplamiento a la salida del circuito donde se inserta la señal de $1kHz$ y se mide la tensión $V_o=1V_{pp}$.
\sangria{} La impedancia de salida se calcula como:
\begin{center}
    \Large
    \[ Z_o=\frac{V_o}{I_o}=\frac{V_o}{\frac{V_g-V_o}{R_S}} \]
    \normalsize %
\end{center}
\midTitle{blue}{Mediciones Obtenidas}{}

\begin{itemize}
    \item $V_o=1Vpp$
    \item $V_g=5,1Vpp$
    \item $Rs=6,8K\Omega$
\end{itemize}
\sangria{} Con los valores obtenidos en las mediciones se tiene el siguiente resultado:
\begin{center}
    \Large
    \[ Z_o=\frac{1V}{\frac{5,1V-1V}{6,8K\Omega}}=1653,54\Omega \]
    \normalsize %
\end{center}
\sangria{} Para el cálculo de los parámetros de pequeña señal se utiliza el modelo híbrido de pequeña señal de la conficuración surtidor común con fuente de corriente:
\imagen[Circuito para análisis de pequeña señal]{15cm}{./imagenes/corrientesenal.png}
\sangria{} 
\sangria{} Ahora se puede despejar el valor da la impedancia de salida del trasnistor $1/h_{oe}$ a partir del valor medido de Zo y el valor calculado de $r_{ds}$:
\begin{center}
    \Large
    \[ Z_o = (\frac{1}{r_{ds}}+\frac{1}{1/h_{oe}})^{-1} \]
    \[ \frac{1}{Z_o} = \frac{1}{r_{ds}}+h_{oe} \]
    \[ h_{oe}=\frac{1}{Z_o}-\frac{1}{r_{ds}}\]
    \[ h_{oe}=438,1\mu S \]
    \[\frac{1}{h_{oe}}=2282,5\Omega \]
    \normalsize %
\end{center}

\subsection{Cálculos experimentales de Zi, Av y Ai}
% poner fotossss

\imagen[Circuito para primera parte de pequeña señal]{15cm}{./imagenes/impeentrada.png}

\sangria{} Para hacer las mediciones de la impedancia de entrada, ganancia de tensión y ganancia de corriente se inserta la señal de $1kHz$ por el capacitor de acoplamiento en la base del transistor $C_i$ y se va aumentando la tensión de la señal de entrada hasta tener una tensión de salida de $V_L=1V_{pp}$.
\sangria{} Para hacer estas mediciones se se agrega una resistencia en serie en la entrada de la base denominada resistencia sensora $R_S=6000\Omega$ Donde se mide la tensión $V_g$ y $V_i$.
\sangria{} 
\sangria{} La impedancia de entrada se calcula como:
\begin{center}
    \Large
    \[ Z_i=\frac{V_i}{I_i}=\frac{V_i}{\frac{V_g-V_i}{R_S}} \]
    \normalsize %
\end{center}
\sangria{} La ganancia de corriente se calcula como:
\begin{center}
    \Large
\begin{equation*}
    A_i=\frac{i_L}{i_i}=\frac{V_L/R_L}{(V_g-V_i)/R_S},     Donde R_L=\frac{1}{h_{oe}}
\end{equation*}
    \normalsize %
\end{center}
\sangria{} La ganancia de tensión se calcula como:
\begin{center}
    \Large
\begin{equation*}
    A_v=\frac{V_L}{V_i} 
\end{equation*}
    \normalsize %
\end{center}

\midTitle{blue}{Mediciones Obtenidas}{}

\begin{itemize}
    \item $V_L=1V_{pp}$
    \item $V_i=22.5mV_{pp}$
    \item $V_g=37.5mV_{pp}$
    \item $R_S=1M\Omega$
\end{itemize}
\sangria{} 
\sangria{} Con los valores obtenidos en las mediciones se tienen los siguientes resultados:  
\begin{center}
    \Large
    \[ Z_i=\frac{22.5mV}{\frac{32.5mV-22.5mV}{1M\Omega}}=1,5M\Omega \]
    \normalsize %
\end{center}
\begin{center}
    \Large
\begin{equation*}
    A_i=\frac{1V/2282,5\Omega}{(37.5mV-22.5mV)/1M\Omega}=29207,7
\end{equation*}
    \normalsize %
\end{center}
\begin{center}
    \Large
\begin{equation*}
    A_v=\frac{V_L}{V_i}=\frac{1V}{22.5mV}=44,44
\end{equation*}
    \normalsize %
\end{center}


