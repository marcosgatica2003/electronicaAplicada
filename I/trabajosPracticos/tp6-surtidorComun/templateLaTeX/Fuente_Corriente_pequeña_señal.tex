\subsection{Cálculo experimental de Zo}

\imagen[Circuito para mediciones de Zo]{10cm}{./imagenes/corrienteZo.png}
\imagen[Placa al realizar las mediciones para Zo]{10cm}{./imagenes/impesalida.jpg}

\sangria{Para medir la impedancia de salida se pasiva la fuente de señal de entrada ($V_i=0$) y se conecta una resistencia sensora ($R_{sensor}$) en serie con el capacitor de acoplamiento. Se inserta una señal de $1\text{ kHz}$ por esta rama y se mide la tensión $V_g$ (antes de la $R_{sensor}$) y $V_o$ (en el drenador), buscando ajustar $V_o=1\text{ V}_{pp}$.}

\sangria{La impedancia de salida se calcula como:}
$$Z_o=\frac{V_o}{I_o}=\frac{V_o}{\frac{V_g-V_o}{R_{sensor}}}$$

\midTitle{blue}{Mediciones Obtenidas (Zo Activa)}{}
\begin{itemize}
    \item $V_o=1\text{ V}_{pp}$
    \item $V_g=5,1\text{ V}_{pp}$
    \item $R_{sensor}=6,8\text{ k}\Omega$
\end{itemize}

\sangria{Con los valores obtenidos en las mediciones se tiene el siguiente resultado:}
$$Z_o=\frac{1\text{ V}}{\frac{5,1\text{ V}-1\text{ V}}{6,8\text{ k}\Omega}} \quad \rightarrow \quad Z_o = \frac{1\text{ V}}{0,6029\text{ mA}} \quad \rightarrow \quad Z_o \approx 1658,5\ \Omega$$

\sangria{Para el cálculo de los parámetros de pequeña señal se utiliza el modelo híbrido de la configuración surtidor común con fuente de corriente:}

\imagen[Circuito para análisis de pequeña señal]{15cm}{./imagenes/corrientesenal.png}

\sangria{Ahora se puede despejar el valor de la impedancia de salida del transistor BJT ($1/h_{oe}$) a partir del valor medido de $Z_o$ y el valor de $r_{ds}$ (calculado en $7200\ \Omega$ en la sección anterior):}
$$Z_o = r_{ds} \parallel \frac{1}{h_{oe}} \quad \rightarrow \quad \frac{1}{Z_o} = \frac{1}{r_{ds}}+h_{oe}$$
$$h_{oe}=\frac{1}{Z_o}-\frac{1}{r_{ds}} \quad \rightarrow \quad h_{oe}=\frac{1}{1658,5\ \Omega}-\frac{1}{7200\ \Omega} \quad \rightarrow \quad h_{oe} \approx 464,1\ \mu S$$
$$\frac{1}{h_{oe}} \approx 2154,7\ \Omega$$


\subsection{Cálculo analítico $A_{i} \ A_{V} \ Z_{i}$ (Carga Activa)}

\sangria{Utilizando los mismos parámetros $g_m$ y $R_G$ calculados para la carga pasiva, pero reemplazando la impedancia de carga $R_D$ por la $Z_o$ de la carga activa (calculada como $Z_o = r_{ds} \parallel 1/h_{oe}$), obtenemos los nuevos valores analíticos.}

\sangria{La impedancia de entrada no cambia:}
$$Z_i = R_G \quad \rightarrow \quad Z_i = \SI{1}{\mega\ohm}$$

\sangria{La impedancia de salida analítica es el paralelo de $r_{ds}$ y $1/h_{oe}$. Usando los valores experimentales/calculados previos ($r_{ds} \approx 7200\ \Omega$ y $1/h_{oe} \approx 2154,7\ \Omega$):}
$$Z_o = r_{ds} \parallel \frac{1}{h_{oe}} \quad \rightarrow \quad Z_o = 7200\ \Omega \parallel 2154,7\ \Omega \quad \rightarrow \quad Z_o \approx 1658,5\ \Omega$$

\sangria{La ganancia de tensión analítica ahora es:}
$$A_v = -g_m \cdot Z_o \quad \rightarrow \quad A_v = -(6,71\text{ mS}) \cdot (1658,5\ \Omega) \quad \rightarrow \quad A_v \approx -11,13$$

\sangria{La ganancia de corriente analítica (usando la misma fórmula que en la etapa pasiva):}
$$A_i = -g_m \cdot R_G \quad \rightarrow \quad A_i = -(6,71\text{ mS}) \cdot (1\text{ M}\Omega) \quad \rightarrow \quad A_i = -6710$$


\subsection{Cálculos experimentales de Zi, Av y Ai (Carga Activa)}
\imagen[Circuito para primera parte de pequeña señal]{8cm}{./imagenes/impeentrada.png}

\sangria{Para hacer las mediciones de la impedancia de entrada, ganancia de tensión y ganancia de corriente se inserta la señal de $1\text{ kHz}$ por el capacitor de acoplamiento en la \textbf{compuerta} del transistor ($C_i$) y se aumenta la tensión de la señal de entrada hasta tener una tensión de salida de $V_L=1\text{ V}_{pp}$.}

\sangria{Para estas mediciones se agrega una resistencia en serie en la entrada de la compuerta, denominada resistencia sensora $R_{sensor}=\SI{1}{\mega\ohm}$. Se mide la tensión $V_g$ (antes de la $R_{sensor}$) y $V_i$ (en la compuerta).}

\sangria{La impedancia de entrada se calcula como:}
$$Z_i=\frac{V_i}{I_i}=\frac{V_i}{\frac{V_g-V_i}{R_{sensor}}}$$

\sangria{La ganancia de corriente se calcula como:}
$$A_i=\frac{i_L}{i_i}=\frac{V_L/R_L}{(V_g-V_i)/R_{sensor}}, \quad \text{donde } R_L = Z_o \approx 1/h_{oe}$$



\sangria{La ganancia de tensión se calcula como:}
$$A_v=\frac{V_L}{V_i}$$

\midTitle{blue}{Mediciones Obtenidas (Zi, Av, Ai Activa)}{}
\begin{itemize}
    \item $V_L=1\text{ V}_{pp}$
    \item $V_i=22.5\text{ mV}_{pp}$
    \item $V_g=37.5\text{ mV}_{pp}$
    \item $R_{sensor}=\SI{1}{\mega\ohm}$
\end{itemize}

\sangria{Con los valores obtenidos en las mediciones se tienen los siguientes resultados. Para $R_L$ se utiliza el valor de $1/h_{oe}$ calculado ($2154,7\ \Omega$):}

$$Z_i=\frac{22.5\text{ mV}}{\frac{37.5\text{ mV}-22.5\text{ mV}}{1\text{ M}\Omega}} \quad \rightarrow \quad Z_i = 1,5\text{ M}\Omega$$

$$A_i=\frac{1\text{ V} / 2154,7\ \Omega}{(37.5\text{ mV}-22.5\text{ mV})/1\text{ M}\Omega} \quad \rightarrow \quad A_i \approx 30924$$

$$A_v=\frac{1\text{ V}}{22.5\text{ mV}} \quad \rightarrow \quad A_v \approx 44,44$$


% --- TABLA COMPARATIVA FINAL ---
\begin{table}[H]
\caption{Comparativa de parámetros de Pequeña Señal (Carga Pasiva vs. Carga Activa)}
\label{tab:comparativa-ac-completa} % ¡Recordá usar un label único!
\resizebox{\textwidth}{!}{%
\begin{tabular}{|
>{\columncolor[HTML]{FFCCC9}}c |
c | c | c | c | c | c |}
\hline
\rowcolor[HTML]{FFFC9E}
\textbf{Parámetro} & \textbf{Calc. (Pasivo)} & \textbf{Medido (Pasivo)} & \textbf{Calc. (Activo)} & \textbf{Medido (Activo)} & \textbf{Desvío Pasivo (calculado vs medido)} & \textbf{Desvío Activo (calculado vs medido)} \\ \hline

% Fila Zi
$Z_i$ & $\SI{1}{\mega\ohm}$ & $1,088\text{ M}\Omega$ & $\SI{1}{\mega\ohm}$ & $1,5\text{ M}\Omega$ & 8,8\% & 50,0\% \\ \hline

% Fila Zo
$Z_o$ & $1681,51\ \Omega$ & $1440\ \Omega$ & $1658,5\ \Omega$ & $1658,5\ \Omega$ & 14,4\% & 1,4\% \\ \hline

% Fila Av
$A_v$ & $-11,28$ & $27,027$ & $-11,13$ & $44,44$ & 340\% & 494\% \\ \hline

% Fila Ai
$A_i$ & $-6710$ & $16339$ & $-6710$ & $30924$ & 343,5\% & 560,8\% \\ \hline
\end{tabular}%
}
\end{table}

% --- SECCIÓN DE RESULTADOS ---
\midTitle{teal}{Resultados}{}

\sangria{Al medir la impedancia de salida $Z_o$ de la configuración con carga activa, fue necesario seleccionar una resistencia sensora ($R_{sensor}$) adecuada. Basándonos en el análisis de la sección anterior, sabíamos que $Z_o$ sería el paralelo entre $r_{ds}$ y $1/h_{oe}$. Habiendo calculado $r_{ds} \approx 7,2\text{ k}\Omega$ (previamente $6000\ \Omega$, pero corregido a $7200\ \Omega$), se seleccionó un valor comercial del mismo orden, $R_{sensor} = 6,8\text{ k}\Omega$, para obtener una buena sensibilidad en la medición.}

\sangria{Los resultados de la tabla comparativa son muy reveladores. La impedancia de salida $Z_o$ medida ($1658,5\ \Omega$) es casi idéntica al valor calculado analíticamente. Sin embargo, el cálculo de la impedancia interna del espejo de corriente arrojó un valor de $1/h_{oe} \approx 2154,7\ \Omega$ ($2,15\text{ k}\Omega$). Este valor \textbf{no es tan alto como se esperaba teóricamente} para una fuente de corriente, siendo incluso inferior a la $r_{ds}$ del JFET. A pesar de esto, este valor sigue siendo superior a la $R_D$ pasiva ($1,8\text{ k}\Omega$).}

\sangria{Esta mayor impedancia de carga de CA ($Z_o \approx 1,66\text{ k}\Omega$ vs $R_D = 1,8\text{ k}\Omega$ no explica el gran aumento de ganancia) es la causa de que la ganancia de tensión ($A_v = 44,44$) y de corriente ($A_i = 30924$) medidas se dispararan, \textbf{duplicando (y más)} las ganancias obtenidas con la carga pasiva. Esto confirma el objetivo de la carga activa: al presentar una impedancia de carga de CA ($1/h_{oe} \approx 2154,7\ \Omega$) mayor que la $R_D$ pasiva ($1,8\text{ k}\Omega$), la ganancia total del amplificador se incrementa drásticamente. Las enormes discrepancias porcentuales con el modelo analítico demuestran la alta sensibilidad del parámetro $g_m$ a las condiciones reales del punto Q.}
