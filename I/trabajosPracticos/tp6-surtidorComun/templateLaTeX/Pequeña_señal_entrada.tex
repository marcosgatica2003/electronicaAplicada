\subsection{Cálculo analítico $A_{i} \ A_{V} \ Z_{i}$}

\imagen[Modelo híbrido de pequeña señal]{15cm}{./imagenes/pequena.png}

\sangria{Para el análisis de pequeña señal, se utiliza el modelo híbrido del JFET. La impedancia de entrada es igual al valor de la resistencia de compuerta:}
$$Z_i = R_G \quad \rightarrow \quad Z_i = \SI{1}{\mega\ohm}$$

\sangria{La impedancia de salida es la resistencia $r_{ds}$ en paralelo con la resistencia de drenaje $R_D$. Como el valor de $r_{ds}$ es muy grande, se puede despreciar su efecto en el circuito. Por lo tanto, la impedancia de salida queda:}
$$Z_o \approx R_D \quad \rightarrow \quad Z_o \approx \SI{1681,51}{\ohm}$$

\sangria{La ganancia de tensión se calcula como:}
$$A_v = -g_m \cdot R_D \quad \text{donde} \quad g_m = \left(-\frac{2I_{DSS}}{V_{GSoff}}\right) \left(1-\frac{V_{GSQ}}{V_{GSoff}}\right)$$
\sangria{Utilizando los valores calculados en la sección anterior ($I_{DSS}=6,650\text{ mA}$, $V_{GSoff}=-1,402\text{ V}$ y $V_{GSQ}=-0,4106\text{ V}$):}
$$g_m = \left(-\frac{2 \cdot 6,650\text{ mA}}{-1,402\text{ V}}\right) \left(1-\frac{-0,4106\text{ V}}{-1,402\text{ V}}\right) = (9,486\text{ mS}) \cdot (1 - 0,2929) \quad \rightarrow \quad g_m \approx 6,71\text{ mS}$$
$$A_v = -(6,71\text{ mS}) \cdot \SI{1681,51}{\ohm} \quad \rightarrow \quad A_v \approx -11,28$$

\sangria{La ganancia de corriente se calcula como la relación entre la ganancia de tensión y las impedancias:}
$$A_i = A_v \cdot \frac{Z_i}{R_D} \quad \rightarrow \quad A_i = -11,28 \cdot \frac{\SI{1}{\mega\ohm}}{\SI{1681,51}{\ohm}} \quad \rightarrow \quad A_i \approx -6708$$

\subsection{Cálculo experimental de ganancia de tensión, corriente e impedancia de entrada}

\imagen[Circuito para realizar las mediciones de pequeña señal]{15cm}{./imagenes/lala.png}

\sangria{Para hacer las mediciones de la impedancia de entrada, ganancia de tensión y ganancia de corriente, se inserta una señal de $1\text{ kHz}$ por el capacitor de acoplamiento en la \textbf{compuerta} del transistor ($C_i$) y se aumenta la tensión de la señal de entrada hasta tener una tensión de salida $V_L=1\text{ V}_{pp}$. $V_L$ se mide en la rama del drenador, sobre $R_D$.}

\sangria{Para estas mediciones se agrega una resistencia en serie en la entrada de la compuerta, denominada resistencia sensora ($R_{sensor}$, $R_s$ en el diagrama de la guía), de $\SI{1}{\mega\ohm}$. En esta configuración se mide la tensión $V_g$ (antes de la resistencia sensora) y $V_i$ (después de la resistencia, en la compuerta).}

\sangria{La impedancia de entrada se calcula como:}
$$Z_i=\frac{V_i}{I_i} = \frac{V_i}{\frac{V_g-V_i}{R_{sensor}}}$$

\sangria{La ganancia de corriente se calcula como:}
$$A_i=\frac{i_L}{i_i} = \frac{V_L/R_L}{(V_g-V_i)/R_{sensor}}$$

\sangria{La ganancia de tensión se calcula como:}
$$A_v=\frac{V_L}{V_i}$$

\midTitle{blue}{Mediciones Obtenidas}{}
\begin{itemize}
    \item $V_L=1\text{ V}_{pp}$
    \item $V_i=37\text{ mV}_{pp}$
    \item $V_g=71\text{ mV}_{pp}$
    \item $R_{sensor}=\SI{1}{\mega\ohm}$
    \item $R_L=R_D \text{ (normalizada)} = \SI{1.8}{\kilo\ohm}$
\end{itemize}

\sangria{Con los valores obtenidos en las mediciones se tienen los siguientes resultados:}
$$Z_i=\frac{37\text{ mV}}{\frac{71\text{ mV}-37\text{ mV}}{1\text{ M}\Omega}} \quad \rightarrow \quad Z_i \approx 1,088\text{ M}\Omega$$

$$A_i=\frac{1\text{ V} / 1800\ \Omega}{(71\text{ mV}-37\text{ mV})/1\text{ M}\Omega} \quad \rightarrow \quad A_i \approx 16339$$

$$A_v=\frac{1\text{ V}}{37\text{ mV}} \quad \rightarrow \quad A_v \approx 27,027$$
