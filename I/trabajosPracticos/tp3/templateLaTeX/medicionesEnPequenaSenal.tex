\newpage
\section{MEDICIONES EN PEQUEÑA SEÑAL DE $Z_i$, $Z_o$, $A_i$ y $A_v$}
    \subsection{Análisis}
    \sangria{} En este apartado se reemplaza al transistor por su modelo equivalente para pequeñas señales.
\imagen[Circuito híbrido equivalente del transistor]{16cm}{./imagenes/circuitoEquivalenteTrans.png}
\begin{multicols}{2}

\begin{itemize}[nosep]
    \item $h_{fe}$: Ganancia de corriente en polarización directa ¡para $h_{fe} = \beta$!
    \ecuacion{\beta = 540}
    \item $h_{ie}$: Impedancia de entrada del transistor.
\end{itemize}

\ecuacion{h_{ie} = h_{fe}\cdot\frac{25mV}{I_{CQ}}}
\ecuacion{h_{ie} = 540 \cdot \frac{25mV}{7,7903mA}}
\ecuacion{\resaltar{h_{ie} = 1,733 k\Omega}}
Calculo de $R_B$:
\ecuacion{R_B = R_2\frac{V_{BB}}{V_{CC}}}
\ecuacion{R_B = 12,31738k\Omega\frac{2,\overline{24}V}{16v}}
\ecuacion{\resaltar{R_B = 1,84138k\Omega}}
\ecuacion{Z_i = \frac{R_B \cdot h_{ie}}{R_B + h_{ie}}}
\ecuacion{Z_i = \frac{1,84138k\Omega\cdot 1,733k\Omega}{1,84138k\Omega + 1,733k\Omega}}
\ecuacion{\resaltar{Z_i = 892,773\Omega}}
\ecuacion{Z_o \approx R_C}
\ecuacion{\resaltar{Z_o \approx 1200\Omega}}
\ecuacion{A_i = -h_{fe}\frac{R_C}{R_C + R_L}\cdot\frac{R_B}{R_B + h_{ie}}}
\ecuacion{A_i =- 540\frac{1,2k\Omega}{1,2k\Omega + 1k\Omega}\cdot\frac{1,84138k\Omega}{1,84138k\Omega+1,733k\Omega}}
\ecuacion{\resaltar{A_i = -151,738}}
\ecuacion{A_v = A_i\cdot\frac{R_L}{Z_i}}
\ecuacion{A_v = -151,738\cdot\frac{1k\Omega}{892,773\Omega}}
\ecuacion{\resaltar{A_v = -169,962}}

    \subsection{Experimental}
    \sangria{} Se coloca el generador de señales mostrado en la siguiente figura, inyectando una señal (sinusoidal en este caso) con una frecuencia de 1 $KHz$ y con una tensión pico a pico de 1V. El objetivo es medir la tensión en antes y después del resistor sensor ($R_S$).\\
    \textbf{Medidas $Z_i$}
    \begin{itemize}[nosep]
        \item $R_S = 1k \Omega$
        \item $v_s = 20mV_{pp}$
        \item $v_i = 13,75 V_{pp}$
        \item $v_L = 1V_{pp}$
    \end{itemize}
    \ecuacion{Z_i = \frac{V_i}{\frac{V_S - V_i}{R_{SENSORA}}}}
    \ecuacion{Z_i = \frac{13,75V}{\frac{20mV - 13,75V}{1k\Omega}}}
    \ecuacion{\resaltar{Z_i = -1001,457 \Omega}}
    \ecuacion{A_V = \frac{V_L}{V_i}}
    \ecuacion{A_V = \frac{1V}{13,75V}}
    \ecuacion{\resaltar{A_V = 0,073}}
    \ecuacion{A_i = \frac{v_L / R_L}{\frac{v_S - v_i}{R_{SENSORA}}}}
    \ecuacion{A_i = \frac{1V / 1k \Omega}{\frac{20mV - 13,75V}{1k\Omega}}}
    \ecuacion{\resaltar{A_i = -0,073}}
\imagen[Osciloscopio $V_i$]{8cm}{./imagenes/osciloscopioViPunto3.jpg}
\imagen[Conexión en placa]{8cm}{./imagenes/conexionPlacaViPunto3.jpg}
\begin{minipage}[t]{\linewidth}
    \textbf{Medidas $Z_o$}
    \begin{itemize}[nosep]
        \item $R_S = 1,2k\Omega$
        \item $v_S = 1V_{pp}$
        \item $v_o = 0,5 V_{pp}$
    \end{itemize}
    \ecuacion{Z_o = \frac{V_o}{\frac{V_S-V_O}{R_S}}}
    \ecuacion{Z_o = \frac{0,5V}{\frac{1V-0,5V}{1,2k\Omega}}}
    \ecuacion{\resaltar{Z_o = 1200 \Omega}}
    \imagen[Osciloscopio $V_o$]{8cm}{./imagenes/osciloscopioVoPunto3.jpg}
\imagen[Conexión en placa]{8cm}{./imagenes/conexionPlacaVoPunto3.jpg}\end{minipage}
\end{multicols}
