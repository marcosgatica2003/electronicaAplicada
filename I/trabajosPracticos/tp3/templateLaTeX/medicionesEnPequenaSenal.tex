\section{Mediciones en pequeña señal de $Z_i$, $Z_o$, $A_i$ y $A_v$}
\begin{multicols}{2}
    \subsection{Análisis}
    \sangria{} En este apartado se reemplaza al transistor por su modelo equivalente para pequeñas señales.
    \subsection{Experimental}
    \sangria{} Se coloca el generador de señales mostrado en la siguiente figura, inyectando una señal (sinusoidal en este caso) con una frecuencia de 1 $KHz$ y con una tensión pico a pico de 1V. El objetivo es medir la tensión en antes y después del resistor sensor ($R_S$).
\end{multicols}
\imagen[Conexión del generador de señales]{12cm}{./imagenes/3.2experimentalmente.png}
