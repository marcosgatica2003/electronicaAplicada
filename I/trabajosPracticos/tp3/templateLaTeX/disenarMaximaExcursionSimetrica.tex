\section{Diseño para máxima excursión simétrica}

\begin{multicols}{2}
    \sangria{} Se tiene el siguiente circuito, consiste en un transistor configurado en base común cuya entrada es una pequeña señal y en la salida se obtiene la misma señal con mayor tensión.
    \sangria{} El circuito se diseñó para obtener la máxima excursión simétrica, dicese el punto donde se obtiene la mayor variación posible de la señal de entrada (o salida) que no provoca recorte ni por saturación ni por corte del transistor, y que se da de manera simétrica respecto al punto de operación (Q). \\[5pt]
    \textbf{Componentes dados:}
    \begin{itemize}[nosep]
        \item $R_e = 180\Omega$
        \item $R_L = 1 K\Omega$
        \item $R_C = 1,2 K\Omega$
    \end{itemize}
    \columnbreak
    \imagen[Gráfico de máxima excursión simétrica]{6cm}{./imagenes/maximaExcusionSimetrica.png}
\end{multicols}

\imagen[Amplificador base común]{14cm}{./imagenes/amplificadorBaseComun.png}

\begin{multicols}{2}
    \sangria{} Los componentes usados para este amplificador y en este trabajo fueron:
    \begin{itemize}[nosep]
        \item $V_{CC} = 15V$
        \item Transistor BC547B ($\beta = 540$)
    \end{itemize}
    \textbf{Cálculo de $R_1$ y $R_2$:}\\[4pt]
\end{multicols}

\subsection{Simulación}

\subsubsection{Valores de diseño}
\subsubsection{Valores normalizados}

\subsection{Construcción del circuito}
