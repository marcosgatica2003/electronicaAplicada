\begin{multicols}{2}
\section{ANÁLISIS Y TRAZADO DE RECTAS DE CARGA}
        \sangria{} Una vez adoptados los valores de los resistores normalizados (de $1/4 W$), se procedió a calcular nuevamente los valores teoricos de:
        \begin{center}
            $V_{CEQ}, I_{CQ}, I_{R1}, I_{R2} y I_{BQ}$
        \end{center}
        para ser comparados con los valores medidos con el multimetro en el punto anterior.
        \sangria{} En este punto también se han trazado rectas de carga de corriente continua y corriente alterna tomando como valores de resistencias los normalizados para reemplazar en las ecuaciones. El objetivo es visualizar gráficamente la excursión simétrica real sin distorsión.

        \subsection{Cálculo rectas de carga}
        \subsubsection{CC}
        \[
        v_{CE} = V_{CC} - I_C (R_C + R_E)
        \]
        Siendo:
            \ecuacion{I_C = -\frac{1}{R_C + R_E} \cdot V_{CE} + \frac{V_{CC}}{R_C + R_E}}

            \ecuacion{\resaltar{i_C = 0 \Rightarrow v_{CE MAX} = V_{CC} = 15v}}
            \ecuacion{\rightarrow i_{CMAX} = \frac{15v}{1,2k\Omega + 180 \Omega}}
            \ecuacion{\resaltar{i_{CMAX} = 10,86mA}}

        \subsubsection{CA}
        \ecuacion{v_{CE} = V_{CC}^\prime - i_C (\frac{R_CR_L}{R_C + R_L})}
        \ecuacion{i_C = -(\frac{R_C + R_L}{R_C \cdot R_L})v_{CE} + V_{CC}^\prime \cdot (\frac{R_C + R_L}{R_C \cdot R_L})}
        \ecuacion{i_C = -(\frac{R_C + R_L}{R_C \cdot R_L})v_{CE} + (V_{CEQMES} + I_{CQ}(\frac{R_CR_L}{R_C + R_L}) \cdot (\frac{R_C + R_L}{R_CR_L}))}
        \ecuacion{v_{CE} = 0 \Rightarrow i_{CMAX} = V_{CC}^\prime\frac{R_C + R_L}{R_CR_L}}
        \ecuacion{i_C = 0 \Rightarrow v_{CEMAX} = V_{CC}^\prime}
        \ecuacion{\rightarrow V_{CC}^\prime = 4,2493V + 7,7903 mA (\frac{1,2k\Omega\cdot1000\Omega}{1000\Omega+1,2k\Omega})}
        \ecuacion{\resaltar{V_{CC}^\prime = 8,4985v}}
\end{multicols}

\subsection{Recta de carga}
\imagen[Rectas de carga]{8cm}{./imagenes/rectaDeCargas.png}

